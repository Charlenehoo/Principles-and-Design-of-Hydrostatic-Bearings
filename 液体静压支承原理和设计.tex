\documentclass[UTF8, 12pt]{ctexbook}
\usepackage{amsmath}

\title{液体静压支承原理和设计}
\author{刘樵}
\date{\today}

\begin{document}
\maketitle
\tableofcontents

\chapter*{主要符号}
\chapter{概论}
\chapter{液压润滑基础知识}
\chapter{平面支承单元的工作原理和性能计算基础}
\section{概述}
\section{单油腔平面油垫}
\section{静压支承的补偿}
\section{有补偿原件的单腔平面油垫的承载能力和刚度}
\section{多油垫支承的承载能力和刚度}
\section{平面油垫单元的承载能力和刚度}
\section{等效平面油垫}

\section{平面支承单元的性能计算基础}
\begin{equation*}
\begin{aligned}
    (r+h)^2 
    &=R^2+e^2-2Re\cos\phi\\
    &=R^2-2Re\cos\phi+e^2\cos^2\phi+e^2-e^2\cos^2\phi\\
    &=(R-e\cos\phi)^2+e^2(1-\cos^2\phi)\\
    &=(R-e\cos\phi)^2+e^2\sin^2\phi
\end{aligned}
\end{equation*}
近似地取为(忽略等号右边第二项):
\begin{equation*}
    r+h\approx R-e\cos\phi
\end{equation*}
即
\begin{equation*}
    h\approx R-r-e\cos\phi
    =h_0-e\cos\phi
    =h_0\left(1-\frac{e}{h_0}\cos\phi\right)
\end{equation*}
或
\begin{equation}
    h\approx h_0(1-\epsilon\cos\phi)
\end{equation}
式中$\epsilon=\frac{e}{h_0}$——位移率或偏心率。
\subsection{等效平面油垫和油膜厚度不均修正系数}
设有一个矩形
%省略
回油量为:
\begin{equation*}
    dQ_b=2\cdot\frac{ph^3}{12\mu b_1}Rd\phi
\end{equation*}
\begin{equation*}
\begin{aligned}
    Q_b
    &=\frac{pD{h_0}^3}{12\mu b_1} 
        \int_{-\theta_e}^{\theta_e}(1-\epsilon\cos\phi)^3d\phi\\
    &=\frac{pD{h_0}^3}{12\mu b_1} 
        \left[
            2\theta_e 
            -6\epsilon\sin\theta_e 
            +3\epsilon^2(\theta_e+\frac{1}{2}\sin2\theta_e) 
            -\frac{2}{3}\epsilon^3\sin\theta_e(2+\cos^2\theta_e)
        \right]
\end{aligned}
\end{equation*}
圆周方向的流量可按封油边中点处的间隙计算,即
\begin{equation*}
    Q_o=\frac{2p(B-b_1){h_0}^3}{12\mu R\theta_1} 
        (1-\epsilon\cos\theta_e)^3
\end{equation*}
当偏心率较小时,如不计$\epsilon$的高次项,则总回油量为:
\begin{equation*}
\begin{aligned}
    Q'
    &=\frac{pD{h_0}^3}{12\mu b_1}
        (2\theta_e-6\epsilon\sin\theta_e)
        +\frac{2p(B-b_1){h_0}^3}{12\mu R\theta_1}
        (1-3\epsilon\cos\theta_e)\\
    &=\frac{pD{h_0}^3}{12\mu b_1}
        \left[
            2\theta_e-6\epsilon\sin\theta_e
            +\frac{b_1(B-b_1)}{R^2\theta_1}(1-3\epsilon\cos\theta_e)
        \right]\\
    &=\frac{pD{h_0}^3}{12\mu b_1}
        \left\{
            \left[
                2\theta_e+\frac{b_1(B-b_1)}{R^2\theta_1}
            \right]
            -3\epsilon
            \left[
                2\sin\theta_e+\frac{b_1(B-b_1)}{R^2\theta_1}\cos\theta_e
            \right]
        \right\}\\
    &=\frac{pD{h_0}^3}{12\mu b_1}
        \left[
            2\theta_e+\frac{b_1(B-b_1)}{R^2\theta_1}
        \right]
        \left[
            1-3\epsilon\frac{
                    2\sin\theta_e+\frac{b_1(B-b_1)}{R^2\theta_1}\cos\theta_e
                }{
                    2\theta_e+\frac{b_1(B-b_1)}{R^2\theta_1}
                }
        \right]
\end{aligned}
\end{equation*}
%省略
(查表):
\begin{equation*}
\begin{aligned}
    Q''
    &=\frac{ph^3}{6\mu}
        \left(
            \frac{B-b_1}{R\theta_1}+\frac{2R\theta_e}{b_1}
        \right)
    =\frac{pR(h_0-\xi e)^3}{6\mu b_1}
        \left[ 
            \frac{b_1(B-b_1)}{R^2\theta_1}+2\theta_e
        \right]\\
    &=\frac{pR{h_0}^3}{6\mu b_1}
        \left[
            \frac{b_1(B-b_1)}{R^2\theta_1}+2\theta_e
        \right](1-\xi\epsilon)^3
\end{aligned}
\end{equation*}
当$\epsilon$较小而不计高次项时,
\begin{equation*}
    Q''=\frac{pD{h_0}^3}{12\mu b_1}
        \left[
            \frac{b_1(B-b_1)}{R^2\theta_1}+2\theta_e
        \right](1-3\xi\epsilon)
\end{equation*}
如果这两个油垫的出油液阻相等,即在相同的油腔压力下出油量相等,
则由$Q'=Q''$的条件必须使系数$\xi$满足下式:
\begin{equation}
    \xi
    =\frac{
            2\sin\theta_e+\frac{b_1(B-b_1)}{R^2\theta_1}\cos\theta_e
        }{
            2\theta_e+\frac{b_1(B-b_1)}{R^2\theta_1}
        }
    =\frac{
        \frac{\sin\theta_e}{\theta_e}
        +\frac{b_1(B-b_1)}{2R^2\theta_1\theta_e}\cos\theta_e
    }{
        1+\frac{b_1(B-b_1)}{2R^2\theta_1\theta_e}
    }
\end{equation}
已知
\begin{equation*}
    \theta_e
    =\theta-\frac{\theta_1}{2}
    =\theta\left(1-\frac{\theta_1}{2\theta}\right)
    =\theta(1-\bar{\theta_1})
\end{equation*}
令
\begin{equation}
    \lambda_p
    =\frac{b_1(B-b_1)}{2R^2\theta_1\theta_e}
    =\frac{
            2b_1B\left(1-\frac{b_1}{B}\right)
        }{
            D^22\theta\frac{\theta_1}{2\theta}\theta_e
        } 
    =\frac{
            \left(\frac{B}{D}\right)^2\frac{b_1}{B}
                \left(1-\frac{b_1}{B}\right)
        }{
            \theta^2\frac{\theta_1}{2\theta}
                \left(1-\frac{\theta_1}{2\theta}\right)
        }
    =\frac{
            \bar{B}^2\bar{b_1}\left( 1 - \bar{b_1} \right)
        }{
            \theta^2\bar{\theta_1}(1-\bar{\theta_1})
        }
\end{equation}
\chapter{向心轴承}
\end{document}